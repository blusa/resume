%--------------------------------------------------------------------------------------------------------------------
%	PACKAGES AND OTHER DOCUMENT CONFIGURATIONS
%--------------------------------------------------------------------------------------------------------------------

\documentclass{resume} 

\usepackage[left=0.75in,top=0.6in,right=0.75in,bottom=0.6in]{geometry}
\usepackage{hyperref}
\hypersetup{colorlinks=true}

\name{Pablo D. Pusiol}
\address{Citizenship: (double) Italian / Argentine. }
\address{pablo.pusiol@gmail.com - mobile: +54 9 (351) 6227364}
\address{Alta Gracia, Cordoba, Argentina (X5186DVD) }

\begin{document}

%--------------------------------------------------------------------------------------------------------------------
%	EDUCATION SECTION
%--------------------------------------------------------------------------------------------------------------------

\begin{rSection}{Academics}

{\bf M. Sc. in Computer Science (2012 - 2014). (2 yrs. program degree as Licenciado)} \\ 
- Institution: FaMAF -- UNC (School of Mathematics, Astronomy and Physics, National University of Córdoba)\\
- Curriculum focus: Remote sensing and satellite imagery processing in collaboration with CONAE, advanced computer vision techniques, and machine learning applications.\\
- Thesis (March 2014): “Deep Learning for Scene Semantics and Activity Recognition in Sports” — applied convolutional neural networks to tennis broadcast footage to automatically extract and classify semantic events. \href{http://repositoriosdigitales.mincyt.gob.ar/vufind/Record/RDUUNC_0aec6058e869999c7224655c3c2c98bf}{Reference}. {\bf Keywords:} Machine Learning, Deep Learning, CNNs. {\it Advisors:} Guido T. Pusiol, PhD (Stanford, US) and Daniel E. Fridlender, PhD (FaMAF--UNC, AR). \\ 
{\it - Average 7.77 / 10.00 - Thesis 10.00 / 10.00}\\

{\bf B. Sc. in Computer Science (2009 - 2011). (3 yrs. program bachelor).} \\ 
- Affiliation: FaMAF - UNC (School of Mathematics, Astronomy and Physics - National University of Cordoba)\\
{\it - Average 7.77 / 10.00}\\


\end{rSection}

%--------------------------------------------------------------------------------------------------------------------
%	PROFESSIONAL EXPERIENCE SECTION
%--------------------------------------------------------------------------------------------------------------------

\begin{rSection}{Professional Experience}

\begin{rSubsection}{\href{https://spinlock.com.ar}{Spinlock SRL} - Cordoba AR}{(May 2017 - Present)}{New Technologies Director}{}
\item Designed and implemented 1D CNNs for inline haploid/diploid corn classification using Spinlock’s state-of-the-art inline NMR sensor for single-seed analysis (up to 70K samples/h).
\\
\item Technical direction of all user and research applications for Spinlock products. Relevant technologies include: C\#, Python, TensorFlow, PyTorch, and .NET Framework.
\\
\item Leading Noctua, an Edge AI spinoff integrating thermal sensors and advanced AI for early fall detection and cognitive decline monitoring in elder care. Currently under consideration by three major University Hospitals in Córdoba, Buenos Aires, and Catalunya. (\href{https://datarock.ai}{{https://datarock.ai}}).
\\
\item Developed an in-line NMR Spectrometer for non-stop measurement of nutritional facts in a food factory. Deployed in Houston, TX for a leading snack manufacturer. \href{https://www.lavoz.com.ar/negocios/una-pyme-cordobesa-desarrollo-un-tomografo-de-papas-para-pepsico}{Link to media (Spanish)}.
\\
\item Developed a novel method for auto-calibration of NMR signals for oil flow and cut sensors using Deep Learning style-transfer algorithms. Oral presentation at \href{http://www.iapg.org.ar/congresos/2019/JIT3/programa.pdf}{3° Jornadas de Integrando el Mundo Físico y el Digital 2019 - (IAPG), Buenos Aires}.

\begin{rSubsection}{Activity Recognition Inc. - Mountain View, CA}{(January 2015 - April 2017)}{Machine Learning Enginer - Colead - Cofounder}{}
\item  Design, development, deployment of a web-based engine for healthcare
	service recommendations and prognosis on elders. This engine is first-of-its-class, and being deployed in several Japanese cities in association with one of the largest Japanese health services provider \href{https://maia.care}{({http://activityrecognition.com)}}.
\\
\item Design, development, deployment of an end-to-end solution for elders
monitoring and alarm triggering at home and nursing-care facilities using deep learning algorithms
in Far Infrared sensors feed. App Thermix \href{https://apps.apple.com/us/app/thermix-for-flir-one/id1042703406}{available at the App Store}, data at
\href{http://thermset.activityrecognition.com}{http://thermset.activityrecognition.com}. This project won Best overall and best
software at FLIR Bring the Heat San Francisco 2015 \href{https://developer.flir.com/past-developer-events/san-francisco-2015-bring-the-heat-hacker-maker-challenge-report/}{(Link to media)}.
\\
\item Design, development and deployment of an automatic video editing social network using Deep Learning. App is free in the App Store under \href{https://itunes.apple.com/tz/app/vdo-share-videos-friends-groups/id971211565?mt=8#}{VDO}. {\bf Technologies:} Objective-C mainly, some Python/Django.
\\

\item Design, development and deployment of a mobile photo app for automatic selection of best pictures using Deep Learning. App is free in the App Store under \href{https://apps.apple.com/us/app/bestie-automatic-camera-and-filters-for-selfies/id953752073}{Bestie}. Using quantization, Bestie was one of the first apps to run locally (iPhone 4) Convolutional Neural Network forward passes of large models (Comparable with AlexNet). {\bf Technologies:} Objective-C mainly, Libccv and C.


\end{rSubsection}

\begin{rSubsection}{INRIA - Nice, France}{(April 2014 - October 2014)}{Research Intern}{}
\item Detecting people in RGB-D data for long term people tracking and events detection using Deep Learning, in the French Institute for Computer Science Research (INRIA). Contribuing to Dem@care (EU joint project for healthcare long term analysis) and Toyota Research Japan.
\end{rSubsection}

\begin{rSubsection}{Spinlock SRL}{(2012 - 2013)}{Scientific Programmer}{}
\item Designed, implemented and tested a self-contained anti-piracy library/protocol for Windows Applications, currently being used in Spinlock's software. The protocol is based on AES and md5 hashing cryptography techniques. (C\#)
\item Designed, implemented and tested a self-contained protocol for auto-discovery of devices and results transmission in an Enterprise network environment, using network and application layer protocols. (C\#,C/C++)
\item Implemented and migrated the USER APPLICATION of NMR Spectrometers from Parallel/Serial interface to USB. (C\#,C/C++)
\item Designed several Process control and Quality Control applications for company use.
\item Worked in the design team of the new user application of Spinlock's NMR spectrometers (NMR4) backend with novel Pulsed Excitation console.
\end{rSubsection}
\begin{rSubsection}{}{}{R\&D Intern}{}
\item Worked on a team which maintained the USER APPLICATIONS currently provided by Spinlock's NMR spectrometers for the determination of the contents of Oil, Moisture and Fatty Acids in food. (C\#)
\item Worked on a team which designed and developed SCIENTIFIC APPLICATIONS for NMR Flow Meters and a Flow Loop operation. (C\#)
\item Implemented Information Security policies and infrastructure for the company. (Windows Server, C\#)
\end{rSubsection}

%---
%\begin{rSubsection}{TEPINTA.COM}{(2012)}{Product Management / Backend Engineer}{}
%\item Co-founded the startup. Designed, implemented and lead the development of the web backend. The application's backend consists mainly of several machine learning algorithms for search and recommendations (collective intelligence), involving supervised and unsupervised such as lineal and logistic univariate regression, TFIDF . (Python, Django, PostgreSQL). \\
%\end{rSubsection}
%\end{rSection}


%--------------------------------------------------------------------------------------------------------------------
%	OPTIONAL SUBJECTS SECTION
%--------------------------------------------------------------------------------------------------------------------

%\begin{rSection}{Optional Subjects}
%    \begin{rSubsection}{INTRODUCTION TO REMOTE SENSING}{}{}{}
%        \item {\it Average 10.0/10.0} - M.Sc. Subject - FaMAF/CONAE - 2013
%        \item Subject Project : AN APPROACH TO AN AUTOMATIC METHOD FOR THE EXTRACTION OF ROAD NETWORKS IN HIGH RESOLUTION REMOTE SENSING IMAGES. Studied different techniques for the development of an automatic road networks extraction method, including classification in different stages using fuzzy logic, neural networks and classic statistics. A vectorization algorithm of the raster image was proposed as well.
%    \end{rSubsection}
    
%    \begin{rSubsection}{STATISTICAL ANALYSIS OF SATELITE IMAGES (IMAGE PROCESSING)}{}{}{}
%        \item {\it Average 10.0/10.0} - M.Sc. Subject - FaMAF/CONAE - 2012
%        \item Subject Project : PROPOSE AND IMPLEMENT NOVEL EDGE DETECTION ALGORITHMS IN IMPULSIVE NOISE AFFECTED IMAGES. Studied popular edge detection algorithms effects in multichannel images affected with impulsive noise. Proposed and implemented new algorithms to work with images with this noise. (C\#, ENVI/IDL, Matlab)
%    \end{rSubsection}
    
%    \begin{rSubsection}{NUCLEAR MAGNETIC RESONANCE (NMR) PRINCIPLES - Spinlock S.R.L. - 2012}{}{}{}
%        \item Lectured at Spinlock facilities. Themes: RF Basics, Excitation, Relaxation, Pulse Sequences, FID, Echos, post Processing Algorithms, Coils, RF Transmitter, RF Receiver, Fixed Magnets vs Electromagnets, A/D - D/A Conversions, etc.
%    \end{rSubsection}
%\end{rSection} 

%--------------------------------------------------------------------------------------------------------------------
%	GIVEN LECTURES SECTION
%--------------------------------------------------------------------------------------------------------------------

%\begin{rSection}{Given Lectures}
%    \begin{rSubsection}{Scene Understanding using Convolutional Neural Networks}{}{}{}
%    \item National University of Cordoba (UNC) - FaMAF 2014
%    \end{rSubsection}
    
%    \begin{rSubsection}{Deep Learning in Computer Vision}{}{}{}
%    \item INRIA Sophia-Antipolis - France 2014
%    \end{rSubsection}

%    \begin{rSubsection}{An approach to an automatic method for road networks extraction}{}{}{}
%    \item National University of Cordoba (UNC) - FaMAF 2013
%    \end{rSubsection}
        
%    \begin{rSubsection}{Formal properties verification in C/C++ code}{}{}{}
%    \item National University of Cordoba (UNC) - FaMAF 2013, attended majority of CS Department of the school.
%    \end{rSubsection}

%    \begin{rSubsection}{State of the art in Digital elevation models}{}{}{}
%    \item CONAE (National Argentine Spatial Agency) - 2013, Dr. Marcelo Scavuzzo, Dr. Oscar Bustos.
%    \end{rSubsection}

%    \begin{rSubsection}{Edge detection algorithms in Impulsive-noise affected multi-channel images - P. Pusiol}{}{}{}
%    \item National University of Cordoba (UNC) - FaMAF 2012, Dr. Oscar Bustos.
%    \end{rSubsection}
%\end{rSection}


%--------------------------------------------------------------------------------------------------------------------
%	ATTENDED COURSES SECTION
%--------------------------------------------------------------------------------------------------------------------

%\begin{rSection}{Attended Courses}
%    \begin{rSubsection}{Acquisition and processing of SAR (Synthetic Aperture Radar) Images}{}{}{}
%    \item CONAE (National Argentine Spatial Agency) - 2013.
%    \end{rSubsection}

%    \begin{rSubsection}{Applications of Cellular Automatons in Remote Sensing}{}{}{}
%    \item FaMAF. Cordoba, Argentina - 2013
%    \end{rSubsection}

%    \begin{rSubsection}{Good coding practices}{}{}{}
%    \item FaMAF / Motorola Solutions, Inc. Cordoba, Argentina - 2011.
%    \end{rSubsection}

%    \begin{rSubsection}{Copyright Property and Software Licenses}{}{}{}
%    \item Catholic University (UCC), Cordoba, Argentina - 2012.
%    \end{rSubsection}
%\end{rSection}

%--------------------------------------------------------------------------------------------------------------------
%	TECHNICAL SKILLS SECTION
%--------------------------------------------------------------------------------------------------------------------
%\begin{rSection}{Technical Skills}
%\item {\bf Programming:} C\#, Python (Tensorflow) (2017 - today). Objective-C (iOS), Python (Django)(2015 - 2017). 
%\item {\bf Project management \& ML:} DVC (Data Version Control), Jupyerlab.
%\item {\bf Interfaces/Libraries:} .NET, Django, PostgreSQL.
%\item {\bf Operating Systems:} iOS, Windows, Linux, OS X, Windows Server.
%\item {\bf Model Checking/Formal Verification:} CBMC, Alloy, SPIN, LTSA.
%\item {\bf Process Management/Version Control:} Feng Office, Asana, SVN, Trac.
%\item {\bf Satelital imaging:} ENVI.
%\end{rSection}



%--------------------------------------------------------------------------------------------------------------------
%	ONLINE COURSES AND OTHER INTERSTS
%--------------------------------------------------------------------------------------------------------------------
\begin{rSection}{Online courses} %\begin{rSection}{Online courses and other interests}

    % MACHINE LEARNING    
    \begin{rSubsection}{Deep Learning Specialization}{Coursera - deeplearning.ai}{by Andrew Ng}{}
    \item The Deep Learning Specialization is designed to prepare learners
to participate in the development of cutting-edge AI technology,
and to understand the capability, the challenges, and the
consequences of the rise of deep learning. Through five
interconnected courses, learners develop a profound knowledge
of the hottest AI algorithms, mastering deep learning from its
foundations (neural networks) to its industry applications
(Computer Vision, Natural Language Processing, Speech
Recognition, etc.). \href{https://coursera.org/share/53f73312283133f99ebcca704ef2009a}{Link to Certificate}.
    \end{rSubsection}

    % MACHINE LEARNING    
    \begin{rSubsection}{Machine Learning}{Coursera - Stanford University}{by Andrew Ng}{}
    \item This course is about 10 weeks long, and it covers different supervised and unsupervised learning algorithms, from the mathematical foundations to matlab implementations of learning and application of them. {\bf Keywords:} Linear Regression, Logistic Regression, Neural Networks, Clustering, SVM .
    \end{rSubsection}

    % RECOMMENDER SYSTEMS
    \begin{rSubsection}{Introduction to Recommender Systems}{Coursera - University of Minnesota}{by Joseph A Konstan, Michael D Ekstrand}{}
    \item This course is fourteen weeks long, and it is structured into eight modules. {\bf Keywords:} Non-personalized recommendations, Content-Based-Recommendations, TFIDF, Vector Based Models 
    \end{rSubsection}

%    % SOCIAL PSYCHOLOGY
%    \begin{rSubsection}{Social psychology}{Coursera - Wesleyan University}{by Scott Plous}{}
%    \item This course was a example-driven approach to understanding people interactions. {\bf Keywords:} Social Perceptions, Persuation, Behavior in different scenarios.
%    \end{rSubsection}

%    % SOCIAL NETWORK ANALYSIS
%    \begin{rSubsection}{Social Network Analysis}{Coursera - University of Michigan}{by Lada Adamic}{}
%    \item {\bf Keywords:} Random network models (Erdos-Reyni, Barabasi-Albert), Communities, Contagion, opinion formation, coordination and innovation.
%    \end{rSubsection}

%    % MARKETING
%    \begin{rSubsection}{An Introduction to Marketing}{Coursera - Wharton School}{}{}
%    \item {\bf Keywords:} Branding, Customer Centricity.
%    \end{rSubsection}
    
\end{rSection}

\end{document}
